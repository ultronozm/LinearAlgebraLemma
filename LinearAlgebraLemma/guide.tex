\documentclass[reqno]{amsart} \input{common.tex}

\title{Proof of a linear algebra lemma, with links to the corresponding Lean 4 code}

\author{Paul D. Nelson}

\date{18 January 2024}

\begin{document}

\maketitle

We formalize in Lean 4 a proof of \cite[Theorem 1.8]{2020arXiv201202187N}:

\begin{theorem}\label{theorem:cnee1j2i90}
  (\verb|MainConcrete'| in \href{MainConcrete.lean}{MainConcrete})

  Let $M_{n}$ denote the space of complex $n \times n$ matrices, included in the space $M_{n+1}$ of $(n+1) \times (n+1)$ matrices as the upper-left block, e.g., for $n = 2$, as
  \[
    \begin{pmatrix}
      \ast & \ast & 0 \\
      \ast & \ast & 0 \\
      0 & 0 & 0
    \end{pmatrix}.
  \]
  Let $\tau$ be an element of $M_{n+1}$ with the property that no eigenvalue of $\tau$ is also an eigenvalue of the upper-left $n \times n$ submatrix $\tau_0$ of $\tau$.  Let $x \in M_{n+1}$ with $[x,\tau] = 0$, where $[a,b] := a b -b a$.  Let $z$ denote the image in $M_{n+1}$ of the identity element of $M_n$, thus $z = \diag(1,\dotsc,1,0)$ with $n$ ones.  Suppose that
  \[
    [x,[z,\tau]] = [y, \tau]
  \]
  for some $y \in M_n$.  Then $x$ is a scalar matrix.
\end{theorem}

In this note, we record a proof and indicate which parts correspond to which Lean files and theorems.

\begin{remark}
  The original proof reduced to a determinental identity \cite[Theorem 17.2]{2020arXiv201202187N} that was verified using some geometric invariant theory.  We decided to formalize a related but more elementary argument noted in \cite[Remark 5.15]{2023arXiv2309.06314}.
\end{remark}

Theorem \ref{theorem:cnee1j2i90} extends (with the same proof) to algebraically closed fields of characteristic $\neq 2$ (\verb|MainConcrete| in \href{MainConcrete.lean}{MainConcrete}).  It generalizes further to commutative rings in which $2$ is a unit, and may be formulated in the language of endomorphisms of finite free modules (see \cite[Remark 5.15]{2023arXiv2309.06314}):

\begin{theorem}\label{theorem:cnee1jt0hq}
  (\verb|MainAbstract| in \href{MainAbstract.lean}{MainAbstract})
  
  Let $R$ be a nontrivial commutative ring in which $2$ is a unit.  Let $V$ be a finite free module over $R$.  Let $\tau \in \End_R(V \times R)$, with projection $\tau_0 \in \End_R(V)$.  Assume that the characteristic polynomials of $\tau$ and $\tau_0$ are coprime.

  Let $x \in \End_R(V \times R)$ with $[x,\tau] = 0$, let $z$ denote the image under the extension-by-zero map $\End_R(V) \rightarrow \End_R(V \times R)$ of the identity endomorphism of $V$ (corresponding to the matrix $\diag(1,\dotsc,1,0)$), and suppose that
  \[
    [x,[z,\tau]] = [y, \tau]
  \]
  for some $y \in \End_R(V)$, extended by zero to an element of $\End_R(V \times R)$.  Then $x$ is a scalar endomorphism.
\end{theorem}
\begin{remark}\label{remark:cnee1noh9i}
  We have formalized only the special case of Theorem \ref{theorem:cnee1jt0hq} in which $R$ is a field.  This case suffices for Theorem \ref{theorem:cnee1j2i90}, which was our motivating goal.  To deduce the general case would require one further formalization, namely that of Lemma \ref{lemma:cnee1m7ggm}, below.
\end{remark}

To verify that the hypotheses of Theorem \ref{theorem:cnee1j2i90} translate to those of Theorem \ref{theorem:cnee1jt0hq}, we use the following:
\begin{lemma}\label{lemma:cnee1nts0y}
  (\verb|coprime_of_disjoint_roots| in \href{CoprimeOfDisjointRoots.lean}{CoprimeOfDisjointRoots})

  Let $R$ be an algebraically closed field.  Let $p, q \in R[X]$ be nonzero polynomials with no common root.  Then $p$ and $q$ are coprime, that is to say, we can write $1 = a p + b q$ with $a,b \in R[X]$.
\end{lemma}

Formalizing the passage from Theorem \ref{theorem:cnee1jt0hq} to \ref{theorem:cnee1j2i90} requires checking that various operations are compatible with the passage from $V \times R$ to $R^{n+1}$ obtained by choosing a basis of $V$.  This is done, laboriously, in \href{MainConcrete.lean}{MainConcrete}.

It remains to explain the proof of (the field case of) Theorem \ref{theorem:cnee1j2i90}.  This requires a couple further lemmas.

\begin{lemma}\label{lemma:cnee1nm8ia}
  (\verb|cyclic_e_of_coprime_charpoly| and \verb|cyclic_e'_of_coprime_charpoly| in \href{CyclicOfCoprime.lean}{CyclicOfCoprime})

  Let $R$ be a nontrivial commutative ring.  Let $V$ be a finite free $R$-module.  Let $\tau \in \End_R(V \times R)$ have the property that its and its projection to $\End_R(V)$ have coprime characteristic polynomials.  Then the vector $(0,1) \in V \times R$ and the dual vector $(0,1)^t \in (V \times R)^*$ are cyclic with respect to $\tau$, i.e., we can write every vector in the respective module as a polynomial in $\tau$ applied to those vectors.
\end{lemma}
\begin{remark}
  For the same reasons as in Remark \ref{remark:cnee1noh9i}, Lemma \ref{lemma:cnee1nm8ia} has only been fully formalized over a field.
\end{remark}
\begin{proof}
  Let us show that $(0,1)^t$ is cyclic; a similar argument gives the other assertion.  We may reduce to the case that $R$ is a field using Lemma \ref{lemma:cnee1m7ggm}, below (that lemma has not yet formalized, but is irrelevant if we restrict from the outset to the field case).

  Let $W \leq (V \times R)^*$ denote the $R[\tau]$-submodule generated by $(0,1)^t$; our task is to show that $W = (V \times R)^*$.

  Let $U \leq V \times R$ denote the coannihilator of $W$ (i.e., the kernel of the natural map $V \times R \rightarrow W^*$).  By duality for finite free modules over fields, it is equivalent to check that $U = 0$.  Since $W$ contains $(0,1)^t$, we see that $U$ is contained in the first summand $V \hookrightarrow V \times R$.  Furthermore, it is clear by construction that $U$ is an $R[\tau]$-submodule of $V \times R$.

  We have reduced to verifying the following (\verb|no_invariant_subspaces_of_coprime_charpoly| in \href{CyclicOfCoprime.lean}{CyclicOfCoprime}):
  \begin{itemize}
  \item if $U$ is an $R[\tau]$-submodule of $V \times R$ contained in the first summand $V \hookrightarrow V \times R$, then $U = 0$.
  \end{itemize}

  To that end, let $u \in U$; we must show that $u = 0$.  Let $\tau_0 \in \End_R(V)$ denote the projection of $\tau$, and let $p, p_0 \in R[X]$ denote the characteristic polynomials of $\tau, \tau_0$.  By hypothesis, we may write $1 = a p_\tau + b p_{\tau_0}$ for some $a, b \in R[X]$.  By evaluating this abstract identity of polynomials on $\tau$ and appealing to the consequence $p(\tau) = 0$ of the Cayley--Hamilton theorem, we see that $u = b(\tau) p_0(\tau) u$.  On the other hand, since $u$ lies in the $R[\tau]$-submodule $U$ of $V \hookrightarrow V \times R$, we see by induction on the degree of $p$ that $p_0(\tau) u = p_0(\tau_0) u$.  By another appeal to Cayley--Hamilton, we have $p_0(\tau_0) = 0$.  It follows that $u = 0$, as required.
\end{proof}

\begin{lemma}
  (\verb|injective_of_cyclic| and \verb|injective_of_cyclic'| in \href{InjectiveOfCyclic.lean}{InjectiveOfCyclic})

  Let $R$ be a nontrivial commutative ring.  Let $V$ be a finite free $R$-module (in practice, this will play the role of the module ``$V \times R$'' appearing above).  Let $\tau \in \End_R(V)$

  \begin{enumerate}[(i)]
  \item\label{enumerate:cnee1ov2p9} If $e \in V$ is cyclic with respect to $\tau$, then the map
    \begin{equation*}
      \left\{ x \in \End_R(V) : [x,\tau] = 0 \right\} \rightarrow V
    \end{equation*}
    \begin{equation*}
      x \mapsto x e
    \end{equation*}
    is injective.
  \item\label{enumerate:cnee1ov375} Dually, if $e^* \in V^*$ is cyclic with respect to $\tau$, then the map
    \begin{equation*}
      \left\{ x \in \End_R(V) : [x,\tau] = 0 \right\} \rightarrow V^*
    \end{equation*}
    \begin{equation*}
      x \mapsto e^* x
    \end{equation*}
    is injective.
  \end{enumerate}
\end{lemma}
\begin{proof}
  We verify \eqref{enumerate:cnee1ov2p9}, as the proof of \eqref{enumerate:cnee1ov375} is similar.  It is enough to show that, for $x \in \End_R(V)$ with $[x,\tau] = 0$ and $x e = 0$, we have $x = 0$.  It is enough to show that $x v = 0$ for each $v \in V$.  By hypothesis, we may write $v = p(\tau) e$ for some $p \in R[X]$.  Since $x$ commutes with $\tau$, we see by induction on the degree of $p$ that it also commutes with $p(\tau)$.  Thus $x v = x p(\tau) e = p(\tau) x e = 0$, as required.
\end{proof}

Having established Lemmas \ref{lemma:cnee1nts0y} and \ref{lemma:cnee1nm8ia}, we are now in position to complete the proof of Theorem \ref{theorem:cnee1jt0hq}.
\begin{proof}[Proof of Theorem \ref{theorem:cnee1jt0hq}]
  (see  \href{MainAbstract.lean}{MainAbstract})
  
  Set $t := [x,z] - y$.  Then $[x,z] = t + y$, and by \ref{theorem:cnee1jt0hq}, we have $[t,\tau] = 0$.

  Recall that $z \in \End_R(V \times R)$ is the extension by zero of the identity endomorphism of the first summand $V$ (in matrix language, this is $\diag(1,\dotsc,1,0)$).  Let $e := (0,1) \in V \times R$ and $e^* := (0,1)^t \in (V \times R)^*$ denote, respectively, the inclusion of the identity element in the second summand and the projection onto the second factor.  We may identify $e e^* := e \otimes e^*$ with the endomorphism of $V \times R$ given by extension by zero of the identity endomorphism of the second summand $R$ (in matrix language, this is $\diag(0,\dotsc,0,1)$).  It follows that $z + e e^*$ is the identity endomorphism of $V \times R$, and in particular, commutes with $x$.  We thus have $[x,e e^*] = - t - y$.  Let us evaluate this last identity on the vectors $e$ and $e^*$.  Recalling that $y$ is the extension by zero of an endomorphism of $V$, we have $y e = 0$ and $e^* y = 0$, hence
  \begin{equation*} [x, ee^*] e = - t e
  \end{equation*}
  and
  \begin{equation*}
    e^* [x, e e^* ] = - e^* t.
  \end{equation*}
  On the other hand, by expanding commutator brackets and using that $e^* e = 1$, we compute that
  \begin{equation*} [x, e e^*] e = x (e e^*) e - (e e^*) x e = (x e) - e (e^* x e) = (x - e^* x e) e,
  \end{equation*}
  where here $e^* x e \in R$ is identified with a scalar endomorphism of $V \times R$, and similarly
  \begin{equation*}
    e^* [x, e e^*] = e^* x (e e^*)  - e^* (e e^*) x = (e^* x e) e^* - e^* x = - e^* (x - e^* x e).
  \end{equation*}
  By Lemmas \ref{lemma:cnee1nts0y} and \ref{lemma:cnee1nm8ia}, the maps defined on the centralizer of $\tau$ given by multiplication against $e$ or $e^*$ are injective.  Since both $x - e^* x e$ and $t$ centralize $\tau$, we deduce from the above four identities that
  \begin{equation*}
    x - e^* x e = - t = - (x - e^* x e),
  \end{equation*}
  which implies that $2 x = 2 (e^* x e)$.  By our hypothesis that $2$ is a unit, we conclude that $x$ is a scalar endomorphism.
\end{proof}

This completes our discussion of the proof of the field case of Theorem \ref{theorem:cnee1jt0hq}, hence that of Theorem \ref{theorem:cnee1j2i90}.

The following would suffice to extend the formalization of Theorem \ref{theorem:cnee1jt0hq} to a general commutative ring in which $2$ is a unit:
\begin{lemma}\label{lemma:cnee1m7ggm}
Let $R$ be a commutative ring.  Let $M$ be a finitely-generated $R$-module, and let $N \leq M$ be a submodule.  Assume that $N + \mathfrak{m} M = M$ for each maximal ideal $\mathfrak{m}$ of $R$.  Then $N = M$.
\end{lemma}
\begin{proof}
  We can check whether $N = M$ after localizing at each maximal ideal $\mathfrak{m}$ of $R$ \cite[Prop 3.9]{MR3525784}, so it suffices to consider the case of a local ring $R$ with maximal ideal $\mathfrak{m}$.  We can then appeal to Nakayama's lemma \cite[Cor 2.7]{MR3525784}.
\end{proof}
\begin{remark}
  Mathlib contains suitable formulations of Nakayama's lemma.  The main outstanding ingredient needed to formalize Lemma \ref{lemma:cnee1m7ggm} is the generalization of \verb|ideal_eq_bot_of_localization| from ideals to modules.
\end{remark}


\bibliography{refs}{} \bibliographystyle{plain}
\end{document}
